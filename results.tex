\documentclass{article}
\usepackage{datatool}
\usepackage{graphicx}
\usepackage{datatool}

\begin{document}

\title{ANUGA applied to the ARR Project 15 Urban Flood Modelling test case 
- Updated using Rain on the Grid}

\maketitle

\section{Introduction}
This is a model of the Pasha Bulker flood (June 2007) in part of Merewether,
Newcastle, Australia. It was originally developed by the UNSW Water Reseach Laboratory (WRL),
and various information on the event is reported in `ARR Project 15: Two
dimensional simulations in urban areas, Representation of buildings in 2D
numerical flood models'. This chapter should be consulted for details on this
test case, and reports on the performance of other models. 

The ANUGA model was initially developed by WRL (who gave us permission to
include it here) and further refined by ANUGA developers to include variable
friction (with roads having a manning's n of 0.02, and everywhere else having
0.04), an inlet operator, and options for either using buildings as topography, or
buildings cut out of the mesh.

The model was then updated using rainfall information provided by WRL in May 2019. 
Permission was granted to use their data, along with permission from Newcastle City Council.

A rain on the grid model was built using the rainfall excess hyetograph taken from
the WBNM model created by WRL. 

ALS data was obtained from here (http://elevation.fsdf.org.au/index.html) and the existing
terrain in the original Merewether calibration was "married" with the ALS data to created a 
model of the entire Merewether catchment for rain on the grid modelling.

The ANUGA developers then refined the manning's roughnesses. The catchment has 4
distinct roughness areas; road were assinged a manning's n of 0.02, maintained parks
a manning's n of 0.035, residential areas a manning's n of 0.15 and forested areas
a manning's n of 0.15. All the manning's n values are in line with numerous adopted flood studies
done my industry experts.

No other alteratons were done to the model other than ensuring the catchment extents were 
inline with the data provided by WRL.
 

\section{Results}
Table~\ref{tab:comparison} compares the observed peak stage during the flood to
ANUGA, and also values for a TUFLOW model which are reported in the ARR Project
15 report. The ANUGA values should be similar to the field observations, and
even more-so to the TUFLOW numerical model values. This suggests both models
are giving a similar solution, with the main source of errors being related to
the input data and the shallow water assumptions, rather than the numerical
scheme.

The recorded level ID2 in the ObservationPoints.csv file is not correct as it is below
the terrain level.

A plot of the hydrograph calculated by WRL using WBNM was also compared against the 
hydrograph generated by ANUGA using rain on the grid. This comparison proves ANUGA's hydrologic
capability. 

The traditional two step methodology being used in the engineering industry is hydrographs generate
with a hydrologic model (such as WBNM or RAFTS) are then used in a 2D hydaulic model.
As the hydrographs have already been routed by the hydrology model, rerouting these hydrographs 
in the 2D hydraulic model causes double routing of the flows, which artificially lowers peak flows 
(and possibly flood levels). 

Another issue that was noticed was in the WBNM file prepared by WRL. For urban catchments the flowpath 
routing factor. The catchment is urbanised and has no natural creek.
The overflow path is a road. The WBNM user manual recommends a value of 0.5. WRL used a factor of 1.0.
The peak WBNM flow as calculated by WRL is approximately 20m3/s.
The amended peak WBNM flow as calculated in this study is approximately 22m3/s. 
The peak flow generated by ANUGA using rain on the grid is approximately 24m3/s.
See Observations directory for the hydrographs.

Figure~\ref{fig:flowrate} shows a plot of the flows at transect 1. Note the similarity 
of the two hydrographs.
\begin{figure}
\includegraphics[width=\textwidth]{flowrate.png}
\label{fig:flowrate}
\end{figure}  

\DTLloaddb{comparison}{Stage_point_comparison.csv}
\begin{table}
\caption{Comparison of peak stage field observations, the ANUGA model, and a
TUFLOW model (developed by WRL). See the ARR project 15 report for more
information}
\label{tab:comparison}
\DTLdisplaydb{comparison}
\end{table}

Figure~\ref{fig:stationary_vel} shows a vector plot of the velocities around some buildings. It should
show a similar pattern to Figure 30 in the ARR Project 15 report.
\begin{figure}
\includegraphics[width=\textwidth]{velocity_stationary.png}
\caption{Velocity vectors around some buildngs at the end of the simulation
(where it should have reached stationary state). Compare with Figure 30 in the ARR Project 15 report (the results should be similar)}
\label{fig:stationary_vel}
\end{figure}

Figure~\ref{fig:transect1} shows a transect of velocity and depth between 2 buildings (Transect
1 in the ARR Report).  It should show a similar pattern to the TUFLOW and
MIKE21 models described in the ARR Report (their Figures 27 and 28).  In this
site the numerical models show some differences to the physical model data,
which may reflect the limitations both of the physical model, and of the
shallow water assumptions (see ARR Report for further discussion).
\begin{figure}
\includegraphics[width=\textwidth]{Transect1.png}
\caption{Velocity and depth along Transect 1. Compare with Figures 27 and 28 
in the ARR Project 15 report (the results should be similar to TUFLOW and
MIKE21)}
\label{fig:transect1}
\end{figure}

Figure~\ref{fig:froude} shows the froude number (as either 0, $(0-1]$ or $>1$).  It
should show a similar pattern to Figure 29 in the ARR report, although the
colors are different. 
\begin{figure}
\center
\includegraphics[width=0.7\textwidth]{froudeNumber.png}
\caption{Froude number (0, $(0-1]$, or $>1$). The largest category area is
zero, the smallest category area is $>1$. Compare with Figure 29 in ARR report
(the patterns should be quite similar).}
\label{fig:froude}
\end{figure}

\end{document}
